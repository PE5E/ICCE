\documentclass{article}[9pt]

%%----PACKS----%%
% misc
\usepackage[margin=0.5in]{geometry}
% math
\usepackage{amsmath,
            amsfonts,
            amssymb}
% images
\usepackage{graphicx}
% code
\usepackage{listings,
            minted}
\usepackage{inconsolata}
%%-------------%%

%%--SETTINGS---%% 
% minted
\usepackage{xcolor}
\usemintedstyle[C++]{emacs}
\definecolor{cppbg}{rgb}{255,255,255}
\newminted[C+++]{C++}{linenos=true, texcl=true, bgcolor=cppbg}
\newminted[bashing]{zsh}

%%--CUSTOM-----%%
\newenvironment{question}
{
\noindent \bf Problem statement. \rm
}


\newenvironment{solution}
{
\noindent \bf Solution. \rm
}

\begin{document}

\begin{flushright}
	\textbf{Regular: Yuqin Cui (s3215040) \\  Regular: Jaap van der Leest\\  Attendee: Timon van der Berg (s1925873) \\ }
\today
\end{flushright}

\begin{center}
\textbf{C++ Course \\
Assignment 3} \\
\end{center}

\section*{Exercise 10}
\begin{question}
Create a program producing a multiplication table.
\end{question}
\begin{solution}

\begin{C+++}
#include <iostream>

// INPUT integer as command line argument
// OUTPUT multiplication table for given number

int main(int argc, char* argv[])
{
    size_t num = std::stoi(argv[1]);

    for (size_t factor = 1; factor != 11; ++factor)
        
        std::cout
                << factor
                << " * "
                << num
                << " = "
                << factor * num
                << "\n";
}
\end{C+++}
\end{solution}

\end{document}

