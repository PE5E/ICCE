\documentclass{article}[10pt]

%%----PACKS----%%
% misc
\usepackage[margin=1in]{geometry}
% math
\usepackage{amsmath,
            amsfonts,
            amssymb}
% images
\usepackage{graphicx}
% code
\usepackage{listings,
            minted}
\usepackage{inconsolata}
%%-------------%%

%%--SETTINGS---%% 
% minted
\usepackage{xcolor}
\definecolor{cppbg}{rgb}{10,240,240}
\newminted[cpp]{C++}{linenos=true, texcl=true, bgcolor=cppbg, tabsize=4 , frame=lines}
\newminted[bashing]{zsh}

%%--CUSTOM-----%%
\newenvironment{question}
{
\noindent \bf Problem statement. \rm
}

\newenvironment{solution}
{
\noindent \bf Solution. \rm
}

\begin{document}

% authors
\begin{flushright}
	\textbf{Regular: Jaap van der Leest\\  Attendee: Timon van der Berg \\ }
\today
\end{flushright}

% title
\begin{center}
\textbf{C++ Course \\
Assignment 4} \\
\end{center}

%%--BEGIN ACTUAL CONTENT--%%
\section*{Exercise 26}

    \begin{question}                    % introduces question: in boldface
    Describe in your own words what `encapsulation' 
    and `data hiding' means, and why these concepts 
    are important when designing classes.
    Provide a small example of a self-defined class 
    illustrating your explanation. 
    Why is the implementation of a class irrelevant to submit?
    \end{question}

\begin{solution}                    % introduces solution: in boldface
Encapsulation means that the data in a class is not directly available outside the class. 
Instead, member functions provide a way to access the data of the class; the class \textit{encapsulates} the data. The data can only be used through the interface, the header file of the class and the functions defined within the header.
In this sense the data is hidden from the outside of the class. "Data hiding" means that no data from within the class can be seen by other code. Not accessed nor altered.

An example of a class interface
    \begin{minted}[linenos=true]{cpp}
    class Vector{
        int d_xval;
        int d_yval;
        public:
            void setXval(int xval);                 // sets d_xval
            void setYval(int yval);                 // sets d_yval

            int const &xval()  const;              // reads d_xval
            int const &yval()  const;              // reads d_yval

            size_t length(int xval, int yval);     // compute length of vector (a,b)
    };
    \end{minted}

Here the data \verb|int d_xval| and \verb|int d_yval| is not accessable from the outside of the class. The member functions \verb|setXval| and \verb|setYval| will assign values to \verb|d_xval|
and \verb|d_yval|. The member functions \verb|xval| and \verb|yval| can read the values of \verb|d_xval| and \verb|d_yval|. 
Finally, a member function \verb|length| is declared, capable of reading and writing of the value. A user who calls \verb|lenth()| need not worry about how the data is saved or how length is computed. In fact, the programmer may change this implementation at any time, for example to make the code faster. If the interface stays the same, the user should not notice. The data variables are also not available to the user, other than by calling the functions that read them.
So everything regarding \verb|d_xval|and \verb|d_yval| will happen inside the class. 

The implementation of a class is irrelevant because only what is described in the public part of the header is available to implement in other code. So the header defines what data and functions of a class can be used by others.
\end{solution}
\section*{Exercise 27}
\section*{Exercise 28}
\section*{Exercise 29}
\section*{Exercise 30} 
\end{document}

